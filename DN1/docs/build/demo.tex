\documentclass[12pt,a4paper]{article}

\usepackage[a4paper,text={16.5cm,25.2cm},centering]{geometry}
\usepackage{lmodern}
\usepackage{amssymb,amsmath}
\usepackage{bm}
\usepackage{graphicx}
\usepackage{microtype}
\usepackage{hyperref}
\usepackage{minted}
\setlength{\parindent}{0pt}
\setlength{\parskip}{1.2ex}

\hypersetup
       {   pdfauthor = {  },
           pdftitle={  },
           colorlinks=TRUE,
           linkcolor=black,
           citecolor=blue,
           urlcolor=blue
       }






\begin{document}



\section{QR razcep zgornje hessenbergove matrike}
Martin Starič

QR razcep zgornje Hessenbergove matrike \texttt{H} je najbolj primeren s pomočjo Givensovih rotacij, saj zanj potrebujemo najmanjše število operacij. Givensove rotacije temeljijo na podlagi rotacije vektorja $v \in R^2$ s pomočjo rotacijske matrike $R = G(i,j,\theta)$. Ko rotacijsko matriko \texttt{R\_i} pomnožimo z $H_i-1$ in dobimo $H_i$ (kjer i označuje v kateri iteraciji smo), bo drugi element, ki smo ga izbrali za $v$ uničen ($= 0$). Ta postopek iteriramo dokler vseh poddiagonalnih elementov ne uničimo. QR razcep pa je nakoncu takšen, da je $H_n = R$ in zmnožek $R^T_1 * R^T_2 * ... * R^T_n = Q$


\begin{minted}[texcomments = true, mathescape, fontsize=\small, xleftmargin=0.5em]{julia}
using Main.DN1
\end{minted}

Sprva izberimo zgornje Hessenbergovo matriko 


\begin{minted}[texcomments = true, mathescape, fontsize=\small, xleftmargin=0.5em]{julia}
H = DN1.Hessenberg([2 7 2;
                3 2 7;
                0 1 4])
\end{minted}
\begin{minted}[texcomments = true, mathescape, fontsize=\small, xleftmargin=0.5em, frame = leftline]{text}
Main.DN1.ZgornjiHessenberg([2.0 7.0 2.0; 3.0 2.0 7.0; 0.0 1.0 4.0])
\end{minted}

Nato izračunamo njen QR razcep s pomočjo Givensovih rotacij, saj je zaradi Hessebergove oblike uporaba Givensovih rotacij najbolj primerna in učinkovita metoda.


\begin{minted}[texcomments = true, mathescape, fontsize=\small, xleftmargin=0.5em]{julia}
Q,R = DN1.qr(H)
\end{minted}
\begin{minted}[texcomments = true, mathescape, fontsize=\small, xleftmargin=0.5em, frame = leftline]{text}
(Main.DN1.Givens([0.5547001962252291 -0.9782400740023413; -0.83205029433784
37 -0.20747616156053694]), [3.6055512754639896 5.547001962252291 6.93375245
2815364; 0.0 4.8198308300986294 -1.3406151977757776; 0.0 0.0 -4.37330856612
8114])
\end{minted}

\subsection{QR iteracija}
QR iteracija je postopek, kjer matriko A spreminjamo v zgornje trikotno tako, da izračunamo $A = QR$ in jo zamenjamo z $A = RQ$, to ponavljamo več iteracij in nazadnje preberemo diagonalne elemente matrike \texttt{A} - ki so lastne vrednosti matrike \texttt{A}. Sledi demonstracija te metode


\begin{minted}[texcomments = true, mathescape, fontsize=\small, xleftmargin=0.5em]{julia}
lastnevrednosti,lastnivektorji = DN1.eigen(H,1000)
\end{minted}
\begin{minted}[texcomments = true, mathescape, fontsize=\small, xleftmargin=0.5em, frame = leftline]{text}
([7.770937597524546, 3.2439316280979136, -3.014869225622463], [0.7835968187
714317 -0.45907575948619295 0.41859941789671284; 0.600513267795882 0.386971
7342630519 -0.6997404462245693; 0.15924773408875592 0.7996888919658832 0.57
89109044179228])
\end{minted}

Da rezultat primerjamo, si pomagamo s knjižnico LinearAlgebra, ki ima funkciji za izračun qr razcepa in lastnih vrednosti


\begin{minted}[texcomments = true, mathescape, fontsize=\small, xleftmargin=0.5em]{julia}
using LinearAlgebra
mt =  [2 7 2;
        3 2 7;
        0 1 4]
QR = LinearAlgebra.qr(mt)
podatki = LinearAlgebra.eigen(mt)
podatki.values
\end{minted}
\begin{minted}[texcomments = true, mathescape, fontsize=\small, xleftmargin=0.5em, frame = leftline]{text}
3-element Vector{Float64}:
 -3.0148692256224603
  3.24393162809791
  7.770937597524555
\end{minted}

Da preverimo čas potreben za izvedbo implementiranih funkcij si pomagamo z dekoratorjem @timed, oglejmo si časovno potratnost metode qr


\begin{minted}[texcomments = true, mathescape, fontsize=\small, xleftmargin=0.5em]{julia}
dn_QR_Result = (@timed DN1.qr(H))
LA_QR_Result = (@timed LinearAlgebra.qr(mt))
dn_QR_Result.time - LA_QR_Result.time
\end{minted}
\begin{minted}[texcomments = true, mathescape, fontsize=\small, xleftmargin=0.5em, frame = leftline]{text}
-1.69e-5
\end{minted}

Opazimo, da je naša funkcija malenkost hitrejša Sedaj preverimo še rezultate za računanje lastnih vrednosti


\begin{minted}[texcomments = true, mathescape, fontsize=\small, xleftmargin=0.5em]{julia}
dn_eigen_Result = (@timed DN1.eigen(H,10))
LA_eigen_Result = (@timed LinearAlgebra.eigen(mt))
dn_eigen_Result.time - LA_eigen_Result.time

dn_eigen_Result2 = (@timed DN1.eigen(H,100))
dn_eigen_Result2.time - LA_eigen_Result.time
\end{minted}
\begin{minted}[texcomments = true, mathescape, fontsize=\small, xleftmargin=0.5em, frame = leftline]{text}
2.3299999999999997e-5
\end{minted}

Izkaže se, da je vse odvisno od željene natančnosti, v kolikor želimo višjo natančnost metode, je naša implementirana funkcija z qr iteracijo počasnejša



\end{document}
